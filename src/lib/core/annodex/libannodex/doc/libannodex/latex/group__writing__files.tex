\section{Writing to files and file descriptors}
\label{group__writing__files}\index{Writing to files and file descriptors@{Writing to files and file descriptors}}
If you wish to write Annodex media to a file or file descriptor (such as a network socket), it can be directly written as follows:

\begin{itemize}
\item open an annodex using {\bf anx\_\-open()}{\rm (p.\,\pageref{anx__general_8h_a3})} or {\bf anx\_\-open\_\-stdio()}{\rm (p.\,\pageref{anx__general_8h_a4})}\item import any media using anx\_\-writer\_\-import()\item call {\bf anx\_\-write()}{\rm (p.\,\pageref{anx__write_8h_a10})} repeatedly until it returns 0 or -1\item close the annodex with {\bf anx\_\-close()}{\rm (p.\,\pageref{anx__general_8h_a7})}\end{itemize}


This procedure is illustrated in src/examples/write-clip-file.c:



\footnotesize\begin{verbatim}
#include <stdio.h>
#include <string.h>

#include <annodex/annodex.h>

int
main (int argc, char *argv[])
{
  ANNODEX * anx = NULL;
  AnxClip my_clip;
  char * infilename, * outfilename;
  long n;

  if (argc != 3) {
    fprintf (stderr, "Usage: %s infile outfile.anx\n", argv[0]);
    exit (1);
  }

  /* Load all importers */
  anx_init_importers ("*/*");

  infilename = argv[1];
  outfilename = argv[2];

  /* Create an ANNODEX* writer, writing to outfilename */
  anx = anx_open (outfilename, ANX_WRITE);

  /* Import infilename into the writer */
  anx_write_import (anx, infilename, NULL /* id */,
                    NULL /* unknown content-type */,
                    0 /* seek_offset */, -1 /* seek_end */, 0 /* flags */);

  /* Insert a clip starting at time 0 */
  memset (&my_clip, 0, sizeof (AnxClip));
  my_clip.anchor_href = "http://www.annodex.net/";
  my_clip.anchor_text = "Find out about Annodex media";

  anx_insert_clip (anx, 0, &my_clip);

  /* End the clip at time 2.0 seconds */
  anx_insert_clip (anx, 2.0, NULL);

  while ((n = anx_write (anx, 1024)) > 0);

  anx_close (anx);

  exit (0);
}
\end{verbatim}
\normalsize
 

