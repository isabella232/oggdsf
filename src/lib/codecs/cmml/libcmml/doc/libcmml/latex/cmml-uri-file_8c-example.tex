\section{cmml-uri-file.c}
{\bf Parse a CMML file given through a file uri (optionally with a fragment offset)}

This example program demonstrates how a CMML file that is given through a file uri and optionally contains a fragment offset can be interpreted. The example can be extended to other schemes such as http and to cover uri queries, too.

The procedure is illustrated in cmml-uri-file.c, which opens a file given through a file uri, and optionally seeks to an offset given the uri fragment specifier. It then prints out the descriptions of all the following clips: 

\footnotesize\begin{verbatim}
#include <stdio.h>

#include <cmml.h>
#include <string.h>

#define BUFSIZE 100000

typedef struct {
  char *scheme;    
  char *authority; 
  char *path;      
  char *querystr;  
  char *fragstr;   
} URI;
 

static URI *
parse_file_uri (const char *uri_string)
{
  const char *location;
  const char *locbegin;
  int length;
  URI *result;
  locbegin = uri_string;
  result = (URI*) calloc(sizeof(URI), sizeof(char));

  /* ignore file:// and authority parts to get path */
  location = strstr (locbegin, "://");
  locbegin = location+3;
  length = strlen(locbegin);
  location = strchr(locbegin, '#'); /* XXX: ignore queries for the moment */
  if (location != NULL) length = location - locbegin;
  result->path = (char*) calloc (length+1, sizeof(char));
  result->path = strncpy(result->path, locbegin, length);
  result->path[length] = '\0';

  if (location != NULL) { 
    /* fragment given */
    length = strlen(location);
    result->fragstr = NULL;
    result->fragstr = (char*) calloc (length, sizeof(char));
    result->fragstr = strncpy(result->fragstr, location+1, length);
  } else {
    result->fragstr = NULL;
  }
  return result;
}

static int
read_clip (CMML * cmml, const CMML_Clip * clip, void * user_data) {
  puts(clip->desc_text);
  return 0;
}

int main(int argc, char *argv[])
{
  char *uri_string = NULL;
  URI * uri;
  CMML * doc;
  long n = 0;

  if (argc < 2) {
    fprintf (stderr, "Usage: %s <file://filename#fragment>\n", argv[0]);
    exit (1);
  }
  uri_string = argv[1];
 
  uri = parse_file_uri(uri_string);

  doc = cmml_open(uri->path);

  /* if fragment given, forward to that */
  if (uri->fragstr != NULL) cmml_skip_to_offset(doc, uri->fragstr);

  cmml_set_read_callbacks (doc, NULL, NULL, read_clip, NULL);

  while (((n = cmml_read (doc, BUFSIZE)) > 0));
  
  cmml_close(doc);

  exit(0);
}
\end{verbatim}
\normalsize
 