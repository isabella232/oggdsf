\section{Fish\-Sound, the sound of fish!}\label{intro}
This is the documentation for the Fish\-Sound C API. Fish\-Sound provides a simple programming interface for decoding and encoding audio data using Xiph.Org codecs (Vorbis and Speex).

libfishsound by itself is designed to handle raw codec streams from a lower level layer such as UDP datagrams. When these codecs are used in files, they are commonly encapsulated in {\tt Ogg} to produce {\em Ogg Vorbis\/} and {\em Speex\/} files. Example C programs using {\tt liboggz} to read and write these files are provided in the libfishsound sources.

For more information on the design and history of libfishsound, see the {\bf About }{\rm (p.\,\pageref{group__about})} section of this documentation, and the {\tt libfishsound} homepage.\subsection{Contents}\label{contents}
\begin{itemize}
\item {\bf fishsound.h }{\rm (p.\,\pageref{fishsound_8h})}: Documentation of the Fish\-Sound API.\end{itemize}


\begin{itemize}
\item {\bf Handling comments }{\rm (p.\,\pageref{comments_8h})}: How to add and retrieve {\em name\/} = {\em value\/} metadata in Vorbis and Speex streams.\end{itemize}


\begin{itemize}
\item {\bf Decoding audio data }{\rm (p.\,\pageref{group__decode})}: How to decode audio data with Fish\-Sound, including source for a fully working Ogg Vorbis and Ogg Speex decoder.\end{itemize}


\begin{itemize}
\item {\bf Encoding audio data }{\rm (p.\,\pageref{group__encode})}: How to encode audio data with Fish\-Sound, including source for a fully working Ogg Vorbis and Ogg Speex encoder.\end{itemize}


\begin{itemize}
\item {\bf Configuration }{\rm (p.\,\pageref{group__configuration})}: Customizing libfishsound to only decode or encode, or to disable support for a particular codec.\end{itemize}


\begin{itemize}
\item {\bf Building }{\rm (p.\,\pageref{group__building})}: Information related to building software that uses libfishsound.\end{itemize}


\begin{itemize}
\item {\bf About }{\rm (p.\,\pageref{group__about})}: Design, motivation, history and acknowledgements.\end{itemize}
\section{Licensing}\label{Licensing}
libfishsound is provided under the following BSD-style open source license:



\footnotesize\begin{verbatim}   Copyright (C) 2003 CSIRO Australia

   Redistribution and use in source and binary forms, with or without
   modification, are permitted provided that the following conditions
   are met:
   
   - Redistributions of source code must retain the above copyright
   notice, this list of conditions and the following disclaimer.
   
   - Redistributions in binary form must reproduce the above copyright
   notice, this list of conditions and the following disclaimer in the
   documentation and/or other materials provided with the distribution.
   
   - Neither the name of the CSIRO nor the names of its
   contributors may be used to endorse or promote products derived from
   this software without specific prior written permission.
   
   THIS SOFTWARE IS PROVIDED BY THE COPYRIGHT HOLDERS AND CONTRIBUTORS
   ``AS IS'' AND ANY EXPRESS OR IMPLIED WARRANTIES, INCLUDING, BUT NOT
   LIMITED TO, THE IMPLIED WARRANTIES OF MERCHANTABILITY AND FITNESS FOR A
   PARTICULAR PURPOSE ARE DISCLAIMED.  IN NO EVENT SHALL THE ORGANISATION OR
   CONTRIBUTORS BE LIABLE FOR ANY DIRECT, INDIRECT, INCIDENTAL, SPECIAL,
   EXEMPLARY, OR CONSEQUENTIAL DAMAGES (INCLUDING, BUT NOT LIMITED TO,
   PROCUREMENT OF SUBSTITUTE GOODS OR SERVICES; LOSS OF USE, DATA, OR
   PROFITS; OR BUSINESS INTERRUPTION) HOWEVER CAUSED AND ON ANY THEORY OF
   LIABILITY, WHETHER IN CONTRACT, STRICT LIABILITY, OR TORT (INCLUDING
   NEGLIGENCE OR OTHERWISE) ARISING IN ANY WAY OUT OF THE USE OF THIS
   SOFTWARE, EVEN IF ADVISED OF THE POSSIBILITY OF SUCH DAMAGE.

\end{verbatim}
\normalsize
 