\section{fishsound.h File Reference}
\label{fishsound_8h}\index{fishsound.h@{fishsound.h}}
The libfishsound C API.  


{\tt \#include $<$fishsound/constants.h$>$}\par
\subsection*{Data Structures}
\begin{CompactItemize}
\item 
struct {\bf Fish\-Sound\-Info}
\begin{CompactList}\small\item\em Info about a particular encoder/decoder instance. \item\end{CompactList}\item 
struct {\bf Fish\-Sound\-Format}
\begin{CompactList}\small\item\em Info about a particular sound format. \item\end{CompactList}\end{CompactItemize}
\subsection*{Typedefs}
\begin{CompactItemize}
\item 
typedef void $\ast$ {\bf Fish\-Sound}
\begin{CompactList}\small\item\em An opaque handle to a Fish\-Sound. \item\end{CompactList}\item 
typedef int($\ast$ {\bf Fish\-Sound\-Decoded} )({\bf Fish\-Sound} $\ast$fsound, float $\ast$$\ast$pcm, long frames, void $\ast$user\_\-data)
\begin{CompactList}\small\item\em Signature of a callback for libfishsound to call when it has decoded audio PCM data. \item\end{CompactList}\item 
typedef int($\ast$ {\bf Fish\-Sound\-Encoded} )({\bf Fish\-Sound} $\ast$fsound, unsigned char $\ast$buf, long bytes, void $\ast$user\_\-data)
\begin{CompactList}\small\item\em Signature of a callback for libfishsound to call when it has encoded data. \item\end{CompactList}\end{CompactItemize}
\subsection*{Functions}
\begin{CompactItemize}
\item 
int {\bf fish\_\-sound\_\-identify} (unsigned char $\ast$buf, long bytes)
\begin{CompactList}\small\item\em Identify a codec based on the first few bytes of data. \item\end{CompactList}\item 
{\bf Fish\-Sound} $\ast$ {\bf fish\_\-sound\_\-new} (int mode, {\bf Fish\-Sound\-Info} $\ast$fsinfo)
\begin{CompactList}\small\item\em Instantiate a new Fish\-Sound$\ast$ handle. \item\end{CompactList}\item 
int {\bf fish\_\-sound\_\-set\_\-decoded\_\-callback} ({\bf Fish\-Sound} $\ast$fsound, {\bf Fish\-Sound\-Decoded} decoded, void $\ast$user\_\-data)
\begin{CompactList}\small\item\em Set the callback for libfishsound to call when it has a block of decoded audio ready. \item\end{CompactList}\item 
int {\bf fish\_\-sound\_\-set\_\-encoded\_\-callback} ({\bf Fish\-Sound} $\ast$fsound, {\bf Fish\-Sound\-Encoded} encoded, void $\ast$user\_\-data)
\begin{CompactList}\small\item\em Set the callback for libfishsound to call when it has a block of encoded data ready. \item\end{CompactList}\item 
long {\bf fish\_\-sound\_\-decode} ({\bf Fish\-Sound} $\ast$fsound, unsigned char $\ast$buf, long bytes)
\begin{CompactList}\small\item\em Decode a block of data. \item\end{CompactList}\item 
long {\bf fish\_\-sound\_\-encode} ({\bf Fish\-Sound} $\ast$fsound, float $\ast$$\ast$pcm, long frames)
\begin{CompactList}\small\item\em Encode a block of audio. \item\end{CompactList}\item 
long {\bf fish\_\-sound\_\-flush} ({\bf Fish\-Sound} $\ast$fsound)
\begin{CompactList}\small\item\em Flush any internally buffered data, forcing encode. \item\end{CompactList}\item 
int {\bf fish\_\-sound\_\-reset} ({\bf Fish\-Sound} $\ast$fsound)
\begin{CompactList}\small\item\em Reset the codec state of a Fish\-Sound object. \item\end{CompactList}\item 
int {\bf fish\_\-sound\_\-delete} ({\bf Fish\-Sound} $\ast$fsound)
\begin{CompactList}\small\item\em Delete a Fish\-Sound object. \item\end{CompactList}\item 
int {\bf fish\_\-sound\_\-command} ({\bf Fish\-Sound} $\ast$fsound, int command, void $\ast$data, int datasize)
\begin{CompactList}\small\item\em Command interface. \item\end{CompactList}\item 
int {\bf fish\_\-sound\_\-get\_\-interleave} ({\bf Fish\-Sound} $\ast$fsound)
\begin{CompactList}\small\item\em Query whether a Fish\-Sound object is using interleaved PCM. \item\end{CompactList}\item 
int {\bf fish\_\-sound\_\-set\_\-interleave} ({\bf Fish\-Sound} $\ast$fsound, int interleave)
\begin{CompactList}\small\item\em Set the PCM format used by a Fish\-Sound object. \item\end{CompactList}\end{CompactItemize}


\subsection{Detailed Description}
The libfishsound C API. 

\subsection{General usage}\label{fishsound_8h_general}
All access is managed via a Fish\-Sound$\ast$ handle. This is instantiated using {\bf fish\_\-sound\_\-new()}{\rm (p.\,\pageref{fishsound_8h_a4})} and should be deleted with {\bf fish\_\-sound\_\-delete()}{\rm (p.\,\pageref{fishsound_8h_a11})} when no longer required. If there is a discontinuity in the input data (eg. after seeking in an input file), call {\bf fish\_\-sound\_\-reset()}{\rm (p.\,\pageref{fishsound_8h_a10})} to reset the internal codec state.\subsection{Decoding}\label{fishsound_8h_decoding}
libfishsound provides callback based decoding: you feed it encoded audio data, and it will call your callback with decoded PCM. A more detailed explanation and a full example of decoding Ogg Vorbis and Speex files is provided in the {\bf Decoding audio data }{\rm (p.\,\pageref{group__decode})} section.\subsection{Encoding}\label{fishsound_8h_encoding}
libfishsound provides callback based encoding: you feed it PCM audio, and it will call your callback with encoded audio data. A more detailed explanation and a full example of encoding Ogg Vorbis and Speex files is provided in the {\bf Encoding audio data }{\rm (p.\,\pageref{group__encode})} section.

\subsection{Typedef Documentation}
\index{fishsound.h@{fishsound.h}!FishSound@{FishSound}}
\index{FishSound@{FishSound}!fishsound.h@{fishsound.h}}
\subsubsection{\setlength{\rightskip}{0pt plus 5cm}typedef void$\ast$ {\bf Fish\-Sound}}\label{fishsound_8h_a0}


An opaque handle to a Fish\-Sound. 

This is returned by fishsound\_\-new() and is passed to all other fish\_\-sound\_\-$\ast$() functions. \index{fishsound.h@{fishsound.h}!FishSoundDecoded@{FishSoundDecoded}}
\index{FishSoundDecoded@{FishSoundDecoded}!fishsound.h@{fishsound.h}}
\subsubsection{\setlength{\rightskip}{0pt plus 5cm}typedef int($\ast$ {\bf Fish\-Sound\-Decoded})({\bf Fish\-Sound} $\ast$ fsound, float $\ast$$\ast$ pcm, long frames, void $\ast$ user\_\-data)}\label{fishsound_8h_a1}


Signature of a callback for libfishsound to call when it has decoded audio PCM data. 

\begin{Desc}
\item[Parameters:]
\begin{description}
\item[{\em fsound}]The Fish\-Sound$\ast$ handle \item[{\em pcm}]The decoded audio \item[{\em frames}]The count of frames decoded \item[{\em user\_\-data}]Arbitrary user data \end{description}
\end{Desc}
\begin{Desc}
\item[Return values:]
\begin{description}
\item[{\em 0}]to continue \item[{\em non-zero}]to stop decoding immediately and return control to the {\bf fish\_\-sound\_\-decode()}{\rm (p.\,\pageref{fishsound_8h_a7})} caller \end{description}
\end{Desc}
\index{fishsound.h@{fishsound.h}!FishSoundEncoded@{FishSoundEncoded}}
\index{FishSoundEncoded@{FishSoundEncoded}!fishsound.h@{fishsound.h}}
\subsubsection{\setlength{\rightskip}{0pt plus 5cm}typedef int($\ast$ {\bf Fish\-Sound\-Encoded})({\bf Fish\-Sound} $\ast$ fsound, unsigned char $\ast$ buf, long bytes, void $\ast$ user\_\-data)}\label{fishsound_8h_a2}


Signature of a callback for libfishsound to call when it has encoded data. 

\begin{Desc}
\item[Parameters:]
\begin{description}
\item[{\em fsound}]The Fish\-Sound$\ast$ handle \item[{\em buf}]The encoded data \item[{\em bytes}]The count of bytes encoded \item[{\em user\_\-data}]Arbitrary user data \end{description}
\end{Desc}
\begin{Desc}
\item[Return values:]
\begin{description}
\item[{\em 0}]to continue \item[{\em non-zero}]to stop encoding immediately and return control to the {\bf fish\_\-sound\_\-encode()}{\rm (p.\,\pageref{fishsound_8h_a8})} caller \end{description}
\end{Desc}


\subsection{Function Documentation}
\index{fishsound.h@{fishsound.h}!fish_sound_command@{fish\_\-sound\_\-command}}
\index{fish_sound_command@{fish\_\-sound\_\-command}!fishsound.h@{fishsound.h}}
\subsubsection{\setlength{\rightskip}{0pt plus 5cm}int fish\_\-sound\_\-command ({\bf Fish\-Sound} $\ast$ {\em fsound}, int {\em command}, void $\ast$ {\em data}, int {\em datasize})}\label{fishsound_8h_a12}


Command interface. 

\begin{Desc}
\item[Parameters:]
\begin{description}
\item[{\em fsound}]A Fish\-Sound$\ast$ handle \item[{\em command}]The command action \item[{\em data}]Command data \item[{\em datasize}]Size of the data in bytes \end{description}
\end{Desc}
\begin{Desc}
\item[Returns:]0 on success, -1 on failure \end{Desc}
\index{fishsound.h@{fishsound.h}!fish_sound_decode@{fish\_\-sound\_\-decode}}
\index{fish_sound_decode@{fish\_\-sound\_\-decode}!fishsound.h@{fishsound.h}}
\subsubsection{\setlength{\rightskip}{0pt plus 5cm}long fish\_\-sound\_\-decode ({\bf Fish\-Sound} $\ast$ {\em fsound}, unsigned char $\ast$ {\em buf}, long {\em bytes})}\label{fishsound_8h_a7}


Decode a block of data. 

\begin{Desc}
\item[Parameters:]
\begin{description}
\item[{\em fsound}]A Fish\-Sound$\ast$ handle (created with mode FISH\_\-SOUND\_\-DECODE) \item[{\em buf}]A buffer of data \item[{\em bytes}]A count of bytes to decode (ie. the length of buf) \end{description}
\end{Desc}
\begin{Desc}
\item[Returns:]The number of bytes consumed \end{Desc}
\index{fishsound.h@{fishsound.h}!fish_sound_delete@{fish\_\-sound\_\-delete}}
\index{fish_sound_delete@{fish\_\-sound\_\-delete}!fishsound.h@{fishsound.h}}
\subsubsection{\setlength{\rightskip}{0pt plus 5cm}int fish\_\-sound\_\-delete ({\bf Fish\-Sound} $\ast$ {\em fsound})}\label{fishsound_8h_a11}


Delete a Fish\-Sound object. 

\begin{Desc}
\item[Parameters:]
\begin{description}
\item[{\em fsound}]A Fish\-Sound$\ast$ handle \end{description}
\end{Desc}
\begin{Desc}
\item[Returns:]0 on success, -1 on failure \end{Desc}
\index{fishsound.h@{fishsound.h}!fish_sound_encode@{fish\_\-sound\_\-encode}}
\index{fish_sound_encode@{fish\_\-sound\_\-encode}!fishsound.h@{fishsound.h}}
\subsubsection{\setlength{\rightskip}{0pt plus 5cm}long fish\_\-sound\_\-encode ({\bf Fish\-Sound} $\ast$ {\em fsound}, float $\ast$$\ast$ {\em pcm}, long {\em frames})}\label{fishsound_8h_a8}


Encode a block of audio. 

\begin{Desc}
\item[Parameters:]
\begin{description}
\item[{\em fsound}]A Fish\-Sound$\ast$ handle (created with mode FISH\_\-SOUND\_\-ENCODE) \item[{\em pcm}]The audio data to encode \item[{\em frames}]A count of frames to encode \end{description}
\end{Desc}
\begin{Desc}
\item[Returns:]The number of frames encoded \end{Desc}
\begin{Desc}
\item[Note:]For multichannel audio, the audio data is interpreted according to the current PCM style \end{Desc}
\index{fishsound.h@{fishsound.h}!fish_sound_flush@{fish\_\-sound\_\-flush}}
\index{fish_sound_flush@{fish\_\-sound\_\-flush}!fishsound.h@{fishsound.h}}
\subsubsection{\setlength{\rightskip}{0pt plus 5cm}long fish\_\-sound\_\-flush ({\bf Fish\-Sound} $\ast$ {\em fsound})}\label{fishsound_8h_a9}


Flush any internally buffered data, forcing encode. 

\begin{Desc}
\item[Parameters:]
\begin{description}
\item[{\em fsound}]A Fish\-Sound$\ast$ handle \end{description}
\end{Desc}
\begin{Desc}
\item[Returns:]0 on success, -1 on failure \end{Desc}
\index{fishsound.h@{fishsound.h}!fish_sound_get_interleave@{fish\_\-sound\_\-get\_\-interleave}}
\index{fish_sound_get_interleave@{fish\_\-sound\_\-get\_\-interleave}!fishsound.h@{fishsound.h}}
\subsubsection{\setlength{\rightskip}{0pt plus 5cm}int fish\_\-sound\_\-get\_\-interleave ({\bf Fish\-Sound} $\ast$ {\em fsound})}\label{fishsound_8h_a13}


Query whether a Fish\-Sound object is using interleaved PCM. 

\begin{Desc}
\item[Parameters:]
\begin{description}
\item[{\em fsound}]A Fish\-Sound$\ast$ handle \end{description}
\end{Desc}
\begin{Desc}
\item[Return values:]
\begin{description}
\item[{\em 0}]{\em fsound\/} uses non-interleaved PCM \item[{\em 1}]{\em fsound\/} uses interleaved PCM \item[{\em -1}]Invalid {\em fsound\/} \end{description}
\end{Desc}
\index{fishsound.h@{fishsound.h}!fish_sound_identify@{fish\_\-sound\_\-identify}}
\index{fish_sound_identify@{fish\_\-sound\_\-identify}!fishsound.h@{fishsound.h}}
\subsubsection{\setlength{\rightskip}{0pt plus 5cm}int fish\_\-sound\_\-identify (unsigned char $\ast$ {\em buf}, long {\em bytes})}\label{fishsound_8h_a3}


Identify a codec based on the first few bytes of data. 

\begin{Desc}
\item[Parameters:]
\begin{description}
\item[{\em buf}]A pointer to the first few bytes of the data \item[{\em bytes}]The count of bytes available at buf \end{description}
\end{Desc}
\begin{Desc}
\item[Return values:]
\begin{description}
\item[{\em FISH\_\-SOUND\_\-xxxxxx}]FISH\_\-SOUND\_\-VORBIS, FISH\_\-SOUND\_\-SPEEX if {\em buf\/} was identified as the initial bytes of a supported codec \item[{\em FISH\_\-SOUND\_\-UNKNOWN}]if the codec could not be identified \item[{\em FISH\_\-SOUND\_\-ERR\_\-SHORT\_\-IDENTIFY}]if {\em bytes\/} is less than 8 \end{description}
\end{Desc}
\begin{Desc}
\item[Note:]If {\em bytes\/} is exactly 8, then only a weak check is performed, which is fast but may return a false positive. 

If {\em bytes\/} is greater than 8, then a stronger check is performed in which an attempt is made to decode {\em buf\/} as the initial header of each supported codec. This is unlikely to return a false positive but is only useful if {\em buf\/} is the entire payload of a packet derived from a lower layer such as Ogg framing or UDP datagrams. \end{Desc}
\index{fishsound.h@{fishsound.h}!fish_sound_new@{fish\_\-sound\_\-new}}
\index{fish_sound_new@{fish\_\-sound\_\-new}!fishsound.h@{fishsound.h}}
\subsubsection{\setlength{\rightskip}{0pt plus 5cm}{\bf Fish\-Sound}$\ast$ fish\_\-sound\_\-new (int {\em mode}, {\bf Fish\-Sound\-Info} $\ast$ {\em fsinfo})}\label{fishsound_8h_a4}


Instantiate a new Fish\-Sound$\ast$ handle. 

\begin{Desc}
\item[Parameters:]
\begin{description}
\item[{\em mode}]FISH\_\-SOUND\_\-DECODE or FISH\_\-SOUND\_\-ENCODE \item[{\em fsinfo}]\end{description}
\end{Desc}
\begin{Desc}
\item[Returns:]A new Fish\-Sound$\ast$ handle, or NULL on error \end{Desc}
\index{fishsound.h@{fishsound.h}!fish_sound_reset@{fish\_\-sound\_\-reset}}
\index{fish_sound_reset@{fish\_\-sound\_\-reset}!fishsound.h@{fishsound.h}}
\subsubsection{\setlength{\rightskip}{0pt plus 5cm}int fish\_\-sound\_\-reset ({\bf Fish\-Sound} $\ast$ {\em fsound})}\label{fishsound_8h_a10}


Reset the codec state of a Fish\-Sound object. 

\begin{Desc}
\item[Parameters:]
\begin{description}
\item[{\em fsound}]A Fish\-Sound$\ast$ handle \end{description}
\end{Desc}
\begin{Desc}
\item[Returns:]0 on success, -1 on failure \end{Desc}
\index{fishsound.h@{fishsound.h}!fish_sound_set_decoded_callback@{fish\_\-sound\_\-set\_\-decoded\_\-callback}}
\index{fish_sound_set_decoded_callback@{fish\_\-sound\_\-set\_\-decoded\_\-callback}!fishsound.h@{fishsound.h}}
\subsubsection{\setlength{\rightskip}{0pt plus 5cm}int fish\_\-sound\_\-set\_\-decoded\_\-callback ({\bf Fish\-Sound} $\ast$ {\em fsound}, {\bf Fish\-Sound\-Decoded} {\em decoded}, void $\ast$ {\em user\_\-data})}\label{fishsound_8h_a5}


Set the callback for libfishsound to call when it has a block of decoded audio ready. 

\begin{Desc}
\item[Parameters:]
\begin{description}
\item[{\em fsound}]A Fish\-Sound$\ast$ handle (created with mode FISH\_\-SOUND\_\-DECODE) \item[{\em decoded}]The callback to call \item[{\em user\_\-data}]Arbitrary user data to pass to the callback \end{description}
\end{Desc}
\begin{Desc}
\item[Returns:]0 on success, -1 on failure \end{Desc}
\index{fishsound.h@{fishsound.h}!fish_sound_set_encoded_callback@{fish\_\-sound\_\-set\_\-encoded\_\-callback}}
\index{fish_sound_set_encoded_callback@{fish\_\-sound\_\-set\_\-encoded\_\-callback}!fishsound.h@{fishsound.h}}
\subsubsection{\setlength{\rightskip}{0pt plus 5cm}int fish\_\-sound\_\-set\_\-encoded\_\-callback ({\bf Fish\-Sound} $\ast$ {\em fsound}, {\bf Fish\-Sound\-Encoded} {\em encoded}, void $\ast$ {\em user\_\-data})}\label{fishsound_8h_a6}


Set the callback for libfishsound to call when it has a block of encoded data ready. 

\begin{Desc}
\item[Parameters:]
\begin{description}
\item[{\em fsound}]A Fish\-Sound$\ast$ handle (created with mode FISH\_\-SOUND\_\-ENCODE) \item[{\em encoded}]The callback to call \item[{\em user\_\-data}]Arbitrary user data to pass to the callback \end{description}
\end{Desc}
\begin{Desc}
\item[Returns:]0 on success, -1 on failure \end{Desc}
\index{fishsound.h@{fishsound.h}!fish_sound_set_interleave@{fish\_\-sound\_\-set\_\-interleave}}
\index{fish_sound_set_interleave@{fish\_\-sound\_\-set\_\-interleave}!fishsound.h@{fishsound.h}}
\subsubsection{\setlength{\rightskip}{0pt plus 5cm}int fish\_\-sound\_\-set\_\-interleave ({\bf Fish\-Sound} $\ast$ {\em fsound}, int {\em interleave})}\label{fishsound_8h_a14}


Set the PCM format used by a Fish\-Sound object. 

The default value is non-interleaved. \begin{Desc}
\item[Parameters:]
\begin{description}
\item[{\em fsound}]A Fish\-Sound$\ast$ handle \item[{\em interleave}]Whether to use interleaved PCM or not. Valid values are 0 for non-interleaved, and 1 for interleaved. \end{description}
\end{Desc}
\begin{Desc}
\item[Return values:]
\begin{description}
\item[{\em 0}]Success \item[{\em -1}]Invalid {\em fsound\/} \end{description}
\end{Desc}
