\section{Building on Win32}
\label{group__win32}\index{Building on Win32@{Building on Win32}}
\subsection{README.Win32}\label{group__win32_win32}


\footnotesize\begin{verbatim}The /Win32 directory contains everything necessary to compile
libfishsound under windows and create:

- libfishsound.dll
- identify.exe
- encode.exe
- decode.exe

Building with the Makefile
==========================

Here's what you need to do:

1) Install libogg.dll, libvorbis.dll, and libspeex.dll.
   (you can get them from http://www.xiph.org/ogg/vorbis/ 
    and http://www.speex.org/ ). encode and decode will
   additionally require libsndfile to be installed (see
   http://www.mega-nerd.com/libsndfile/).

2) cd win32 subdirectory

3) Use "nmake" to create library and executable.

4) Install libfishsound.dll and the include files from the
   /include/fishsound/ directory.

5) You can test the library with the "identify.exe" program
   also found in the win32 subdirectory.


Visual Studio.NET 2003 Installation
===================================

IMPORTANT: The solution files were built for VS.NET 2003 and can't be
opened by VS.NET 2002. If you use VS.NET 2002 you should use the VS6
workspace files and they will be automatically converted to the new
format.

Also included in the solution is a setup and deployment project that
will package and create an installer for the binary output of this
project.

You will need to install the libogg, libvoribs, and libspeex libraries
separately. If you wish to create a debug build and wish to have the
debugger load the symbols for these libraries, you should also include
their project source files. By default the compiler and linker will
look for them in the same parent directory as libfishsound. However
you can install them wherever you choose and must modify the include
directories for the linker and compiler in your projects properties.


Visual Studio Version 6 Installation
====================================

There is also a Visual C++ V 6.0 workspace file in the Win32 directory
in case you have not upgraded to Visual Studio.NET.
\end{verbatim}
\normalsize
 